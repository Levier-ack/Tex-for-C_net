\documentclass{article} %article 文档
\usepackage{enumerate}
\usepackage{pgfplots}
\usepackage{listings}
\usepackage{pythonhighlight}
\usepackage{xcolor}
\usepackage{graphicx}
\usepackage{ctex}  %使用宏包(为了能够显示汉字)
\title{计算机网络实验:第四周实验}  %文章标题
\author{秦宏 2016K8009929009}   %作者的名称
\date{\today}       %日期
% 设置页面的环境,a4纸张大小,左右上下边距信息
\usepackage[a4paper,left=10mm,right=10mm,top=15mm,bottom=15mm]{geometry}  
\pgfplotsset{compat=1.9,width=15cm}  
 
\begin{document}
\lstset{
	keywordstyle= \color{ blue!70},
	commentstyle= \color{red!50!green!50!blue!50}, 
	frame=shadowbox, % 阴影效果
	rulesepcolor= \color{ red!20!green!20!blue!20} ,
	escapeinside=``, % 英文分号中可写入中文
	xleftmargin=1em,xrightmargin=1em, aboveskip=1em
}
	\maketitle          %添加这一句才能够显示标题等信息
	\begin{section}{交换机转发实验}
		\begin{subsection}{实验内容}
			\begin{enumerate}[1)]
				\item 实现对数据结构mac\_port\_map的所有操作,以及数据包的转发和广播操作。
					\begin{lstlisting}[language={[ANSI]C}]
iface_info_t* lookup_port(u8 mac[ETH_ALEN]);
void insert_mac_port(u8 mac[ETH_ALEN], iface_info_t* iface);
int sweep_aged_mac_port_entry();
void broadcast_packet(iface_info_t* iface, const char* packet, int len);
void handle_packet(iface_info_t* iface, char* packet, int len);
					\end{lstlisting}
				\item 使用iperf和给定的拓扑进行实验,对比交换机转发与集线器广播的性能。	
			\end{enumerate}
			\end{subsection}
		\begin{subsection}{实验流程}
			\begin{enumerate}[1)]
				\item 实现对数据结构mac\_port\_map的所有操作,和数据包的转发和广播操作,保证能够ping通。
				\item 分别在s1先后运行hub和switch,使用iperf发送数据包测试传播速率。
				\item 在h1,h2,h3均执行iperf -s后两个节点同时发送数据包,测试传播速率。
			\end{enumerate}
			\end{subsection}
		\begin{subsection}{实验结果}
		首先是switch和hub在单独发生一个数据包发送时的传输速率,由于只有一个传输方向,因此hub与switch的带宽利用率相近,均在0.94左右,差别不大。\vspace{3pt}
		
		然后是switch和hub运行时,选择两个端点同时向对方发送数据包,进行测试,首先是h2和h3相互发送数据包,实验结果如图1和图2,
		\begin{figure}[hp]	
			\centering
			\includegraphics[width=\linewidth,scale=0.1]{hub_23.eps}	
			\caption{hub节点下h2和h3互发数据}
		\end{figure}
		\begin{figure}[hp]	
			\centering
			\includegraphics[width=\linewidth,scale=0.1]{switch_23.eps}	
			\caption{switch节点下h2和h3互发数据}
		\end{figure}
		由图1看到hub节点下,h2和h3互发数据时的带宽利用率为0.938和0.939.图2看到switch节点下,h2和h3互发数据时的带宽利用率为0.937和0.938,看到在hub节点或者switch节点的传输利用率是一样的,没有发现hub和switch的优劣。\vspace{3pt}
		
		然后选择了h1和h3节点相互发送数据包,实验结果如图3和图4,
		\begin{figure}[hp]	
			\centering
			\includegraphics[width=\linewidth,scale=0.1]{hub_13.eps}	
			\caption{hub节点下h1和h3互发数据}
		\end{figure}
		\begin{figure}[hp]	
			\centering
			\includegraphics[width=\linewidth,scale=0.1]{switch_13.eps}	
			\caption{switch节点下h1和h3互发数据}
		\end{figure}
		由图3看到hub节点下,h1和h3互发数据时带宽利用率为0.499和0.495,图4看到switch节点下,h1和h3互发数据时的带宽利用率为0.945和0.905.(考虑到线路的带宽上限与线路带宽的最小值相等计算得到)。\vspace{3pt}
		
		然后进行了h1向h2、h3向h1同时发送数据包的测试,测试结果如图5,图6所示,
		\begin{figure}[hp]	
			\centering
			\includegraphics[width=\linewidth,scale=0.1]{hub_1-2_3-1.eps}	
			\caption{hub节点下1-2和3-1发数据}
		\end{figure}
		\begin{figure}[hp]	
			\centering
			\includegraphics[width=\linewidth,scale=0.1]{switch_1-2_3-1.eps}	
			\caption{switch节点下1-2和3-1互发数据}
		\end{figure}
		由图5看到hub节点下,h1节点发送数据的带宽利用率为0.366,h3节点发送数据的带宽利用率为0.645,但是这个差值的原因还无法得知,我尝试将h1的1带宽调整到10或者比20大,结果仍然是h1带宽为h3的一半。还不能得知原因。\vspace{3pt}
		
		最后进行了sub和switch条件下h1向h2、h3发送数据以及h2、h3向h1发送数据的测试,测试结果如图7,图8,图9所示,
		\begin{figure}[hp]	
			\centering
			\includegraphics[width=\linewidth,scale=0.1]{hub_2-1_3-1.eps}	
			\caption{hub节点下2、3向1发数据}
		\end{figure}
		\begin{figure}[hp]	
			\centering
			\includegraphics[width=\linewidth,scale=0.1]{hub1-2_1-3.eps}	
			\caption{hub节点下1向2、3发数据}
		\end{figure}
		\begin{figure}[h]	
			\centering
			\includegraphics[width=\linewidth,scale=0.1]{switch_1-2_1-3.eps}	
			\caption{switch节点下1向2、3发数据}
		\end{figure}
		由图7看到hub节点下2、3同时向1发送数据的带宽不受影响,因为1的带宽刚好容纳两条数据通路大小,但是图8中1同时向2、3发送时会出现带宽显著下降的情况,显然是广播条件下3的链路上有目标地址为2的数据包占用线路,因此带宽减半,到了图9中1向2、3发送消息时因为switch的选择性保证了2的数据包只会到2的线路上,因此3的带宽不受影响,可见交换机的性能高于集线器。\\
		\end{subsection}
		\begin{subsection}{实验分析}
			\begin{enumerate}[1)]
				\item hub会出现带宽下降的原因是hub是广播数据包的,当两个服务器同时发送数据包到hub时,hub将二者的数据包合并到另一条未传过来的链路上,图3中为h2,此时h2带宽上限为10M,相当于是h1和h3的带宽之和变成了10M,相应的带宽就会下降,反映到图5也是如此,都会因h2带宽限制而下降,而图1的hub之所以未变化是因为h1的带宽限制本身就是20M。
				\item switch则会根据端口信息选择发送端口,这样数据包不会因为集中到一条无需到达的链路从而降低了带宽。结果如秃2,4,6,带宽基本不变。
				\item 代码实现
					\begin{lstlisting}[language={[ANSI]C}]
iface_info_t *lookup_port(u8 mac[ETH_ALEN])//查找函数
{
  int hash_value = hash8(mac, ETH_ALEN);

  mac_port_entry_t *entry = NULL;
  mac_port_entry_t *p = NULL;

  // pthread_mutex_lock(&mac_port_map.lock);

  list_for_each_entry_safe(entry, p, &mac_port_map.hash_table[hash_value], list){
	if (com_str(entry->mac, mac) == 0){
		entry->visited = time(NULL);//刷新时间
		return entry->iface;
	}	
  }
  // pthread_mutex_unlock(&mac_port_map.lock);
  // TODO: implement the lookup process here
  fprintf(stdout, "TODO: implement the lookup process here.\n");
  return NULL;
}
					\end{lstlisting}
					\begin{lstlisting}[language={[ANSI]C}]
// insert the mac -> iface mapping into mac_port table
void insert_mac_port(u8 mac[ETH_ALEN], iface_info_t *iface)//插入函数
{
  int temp = hash8(mac, ETH_ALEN);

  fprintf(stdout, "insert new mac_port\n");
  mac_port_entry_t *ptr = (mac_port_entry_t *)malloc(sizeof(mac_port_entry_t));
  bzero(ptr, sizeof(mac_port_entry_t));//为新节点初始化
  init_list_head(&(ptr->list));
  ptr->visited = time(NULL);
  memcpy(ptr->mac, mac, ETH_ALEN*sizeof(u8));
  fprintf(stdout, "test memcpy\n");
  //memcpy(ptr->iface, iface, sizeof(iface_info_t));
  ptr->iface = iface;

  // pthread_mutex_lock(&mac_port_map.lock);	
  list_add_head(&ptr->list, &mac_port_map.hash_table[temp]);//插入新节点		
  // pthread_mutex_unlock(&mac_port_map.lock);
  // TODO: implement the insertion process here
  fprintf(stdout, "TODO: implement the insertion process here.\n");
}
					\end{lstlisting}
					\begin{lstlisting}[language={[ANSI]C}]
int sweep_aged_mac_port_entry()//老化函数
{
  int count = 0;
  mac_port_entry_t *entry = NULL;
  mac_port_entry_t *p = NULL;
  time_t now = time(NULL);

  pthread_mutex_lock(&mac_port_map.lock);
  for (int i = 0; i < HASH_8BITS; i++){
	list_for_each_entry_safe(entry, p, &mac_port_map.hash_table[i], list){
	  if ((int)(now - entry->visited) > MAC_PORT_TIMEOUT){
	    fprintf(stdout, "TODO: implement the sweeping process here.\n");
	    list_delete_entry(&entry->list);
	    count++;
	    free(entry);
	  }
	}
  }// TODO: implement the sweeping process here
  pthread_mutex_unlock(&mac_port_map.lock);

  return count;
}
					\end{lstlisting}
					\begin{lstlisting}[language={[ANSI]C}]
void handle_packet(iface_info_t *iface, char *packet, int len)
{
  struct ether_header *eh = (struct ether_header *)packet;
  log(DEBUG, "the dst mac address is " ETHER_STRING "
  	.\n", ETHER_FMT(eh->ether_dhost));
  
  iface_info_t *iface_ptr = lookup_port(eh->ether_dhost);//查找目标mac地址

  // dump_mac_port_table();

  if (iface_ptr != NULL){
	fprintf(stdout, "find and send packet\n");
	iface_send_packet(iface_ptr, packet, len);
  }
  else{
	broadcast_packet(iface, packet, len);
  }

  iface_ptr = lookup_port(eh->ether_shost);

  if (iface_ptr != NULL){
	;
  }
  else{
	insert_mac_port(eh->ether_shost, iface);//插入源mac地址
  }
  // TODO: implement the packet forwarding process here
  fprintf(stdout, "TODO: implement the packet forwarding process here.\n");
}
					\end{lstlisting}
			\end{enumerate}
		\end{subsection}
		\begin{subsection}{实验总结}
			此次实验难度在于数据结构的理解,一开始我找不到mac端口存放接收mac地址和发送mac地址的存储位置,之后想明白了mac结构体不考虑接收和发送,只需要存储mac地址,这样代码就会简单许多。\\
			另外这次实现insert函数时用到了最近刚刚温习的malloc函数,加深了对该函数的使用方法条件等的理解。
		\end{subsection}
	\end{section}
\end{document}
