\documentclass{article} %article 文档
\usepackage{enumerate}
\usepackage{pgfplots}
\usepackage{listings}
\usepackage{pythonhighlight}
\usepackage{xcolor}
\usepackage{caption}
\usepackage{subfigure}
\usepackage{graphicx}
\usepackage{ctex}  %使用宏包(为了能够显示汉字)
\title{计算机网络实验:第五周实验}  %文章标题
\author{秦宏 2016K8009929009}   %作者的名称
\date{\today}       %日期
% 设置页面的环境,a4纸张大小,左右上下边距信息
\usepackage[a4paper,left=10mm,right=10mm,top=15mm,bottom=15mm]{geometry}  
\pgfplotsset{compat=1.9,width=15cm}  
 
\begin{document}
\lstset{
	keywordstyle= \color{ blue!70},
	commentstyle= \color{red!50!green!50!blue!50}, 
	frame=shadowbox, % 阴影效果
	rulesepcolor= \color{ red!20!green!20!blue!20} ,
	escapeinside=``, % 英文分号中可写入中文
	xleftmargin=1em,xrightmargin=1em, aboveskip=1em
}
	\maketitle          %添加这一句才能够显示标题等信息
	\begin{section}{生成树机制实验}
		\begin{subsection}{实验内容}
			\begin{enumerate}[1)]
				\item 基于已有代码,实现生成树运行机制,对于给定拓扑(four\_node\_ring.py),计算输出相应状态下的最小生成树拓扑。
				\item 自己构造一个不小于7个节点、冗余链路不少于2条的拓扑,节点和端口的命名规则可参考four\_node\_ring.py,使用stp程序计算输出最小生成树拓扑。	
			\end{enumerate}
			\end{subsection}
		\begin{subsection}{实验流程}
			\begin{enumerate}[1)]
				\item 第一步阅读了主要的代码框架,结合实验内容决定实验中要完成的代码部分。下面是对代码框架的简要概述。
					\begin{enumerate}[(1)]
						\item stp初始化时除了初始化节点基本信息外,还建立了双线程函数stp\_timer\_routine用来完成节点向外周期性发送config消息的任务,其中包括了判断config消息是否过期,并执行发送config消息的函数。在发送时进行判断,只有根节点的指定端口才能够发送config消息,因此省去了对hello定时器的停止操作。
						\item 代码框架中的数据报处理函数stp\_handle\_config\_packet输入了从packet修改而来的config消息,因此stp\_handle\_config\_packet函数主要进行的就是config消息的判断处理步骤。
						\item							stp\_handle\_config\_packet函数的操作步骤如下图所示\\
						\begin{figure}[!b]	
							\centering
							\includegraphics[height=9cm]{routine.png}	
						\end{figure}
 
					\end{enumerate}
				\item 然后就是函数stp\_handle\_config\_packet函数具体实现:
					\begin{enumerate}[(1)]
						\item 网络字节序转换使用了ntohll、ntohl、ntohs三个函数,根据对应变量的字节长度使用,代码如下\\
						\begin{lstlisting}[language={[ANSI]C}]
u64 config_root_id = ntohll(config->root_id);
u32 config_root_path_cost = ntohl(config->root_path_cost);
u64 config_switch_id = ntohll(config->switch_id);
u16 config_port_id = ntohs(config->port_id);
						\end{lstlisting}
						\item config消息的优先级比较,使用的是CfgIsHighPrty函数,它输入上述转化的值和p指针,输出bool变量,若为true,则收到的config消息优先级高,代码如下\\
						\begin{lstlisting}[language={[ANSI]C}]
bool CfgIsHighPrty(u64 config_root_id, u32 config_root_path_cost, 
u64 config_switch_id, u16 config_port_id, stp_port_t *p)
{
if (config_root_id != p->designated_root)
  return p->designated_root > config_root_id;
else if (config_root_path_cost != p->designated_cost)
  return p->designated_cost > config_root_path_cost;
else if (config_switch_id != p->designated_switch)
  return p->designated_switch > config_switch_id;
else if (config_port_id != p->designated_port)
  return p->designated_port > config_port_id;

return false;
}
						\end{lstlisting}
						优先级比较采用的是顺序比较,按根节点ID、根节点开销、上一跳节点、上一跳端口的顺序执行。
						\item 返回得到bool变量值,若为true,执行stp\_port\_send\_config(p)发送config消息,否则执行节点状态更新等一系列步骤。
						\item 替换config消息代码如下
						\begin{lstlisting}[language={[ANSI]C}]
p->designated_root = config_root_id;
p->designated_cost = config_root_path_cost;
p->designated_switch = config_switch_id;
p->designated_port = config_port_id;
						\end{lstlisting}
						\item 更新状态节点要求找到根端口,这里调用函数HighPrtyPort,该函数返回根端口在stp中端口数组中的序号,若返回-1表示找不到根端口,代码实现如下\\
						\begin{lstlisting}[{language={[ANSI]C}}]
int HighPrtyPort()//返回-1时找不到根端口,返回大于等于0的值表示找到根端口
{
int i = 0;
int high_port_id = 0;
bool is_designated = stp_port_is_designated(&stp->ports[0]);

while (is_designated && i < stp->nports){//使i为第一个不是指定端口的序号
  i++;
  is_designated = stp_port_is_designated(&stp->ports[i]);
}

if (i == stp->nports)//判断是否为根节点
  return -1;

stp_port_t* high_port_ptr = &stp->ports[i];
stp_port_t* temp_ptr = NULL;

for ( ; i < stp->nports; i++){
  temp_ptr = &(stp->ports[i]);		
  is_designated = stp_port_is_designated(temp_ptr);

  if (is_designated){
	continue;
  }

  if (high_port_ptr->designated_root != temp_ptr->designated_root){
	if (high_port_ptr->designated_root > temp_ptr->designated_root){
	  high_port_ptr = &stp->ports[i];
	  high_port_id = i;
	  continue;
	}
  }
  else if (high_port_ptr->designated_cost != temp_ptr->designated_cost){
	if (high_port_ptr->designated_cost > temp_ptr->designated_cost){
	  high_port_ptr = &stp->ports[i];
      high_port_id = i;
	  continue;
	}
  }
  else if (high_port_ptr->designated_switch != temp_ptr->designated_switch){
	if (high_port_ptr->designated_switch > temp_ptr->designated_switch){
	  high_port_ptr = &stp->ports[i];
	  high_port_id = i;
	  continue;
	}
  }
  else if (high_port_ptr->designated_port != temp_ptr->designated_port){
  	if (high_port_ptr->designated_port > temp_ptr->designated_port){
	  high_port_ptr = &stp->ports[i];
	  high_port_id = i;
	  continue;
	}
  }
}
return high_port_id;
}
						\end{lstlisting}
						该函数目的是找到所有端口中除指定端口外优先级最高的端口,因此遍历时要排除掉指定端口,首先找到第一个不是指定端口的端口,然后在遍历过程中遇到指定端口直接跳过,从而找出优先级最大的非指定端口,优先级比较依然采取同上面优先级一致的顺序比较,最后返回端口在数组中的id号。\\
						获得根端口后判断是否存在根端口,代码实现如下\\
						\begin{lstlisting}[{language={[ANSI]C}}]
int port_id = HighPrtyPort();
if (port_id == -1){
  stp->root_port = NULL;
  stp->designated_root = stp->switch_id;
  stp->root_path_cost = 0;
}
else {
  stp_port_t* temp = &stp->ports[port_id];
  stp->root_port = temp;
  stp->designated_root = temp->designated_root;
  stp->root_path_cost = temp->designated_cost + temp->path_cost; 
}
						\end{lstlisting}
						如果没有根端口,则该节点为根节点,修改节点消息,如果有根端口,则将该端口的消息更新到节点消息中。
						\item 下一步更新剩余端口的config消息,调用UpdatePort函数,代码实现如下\\
						\begin{lstlisting}[language={[ANSI]C}]
void UpdatePort()
{
  int i = 0;
  bool is_designated;
  stp_port_t* temp_ptr = NULL;
  for (i = 0; i < stp->nports; i++){
	temp_ptr = &stp->ports[i];
	is_designated = stp_port_is_designated(temp_ptr);
	if (is_designated){//是指定端口
	  temp_ptr->designated_root = stp->designated_root;
	  temp_ptr->designated_cost = stp->root_path_cost;
	}
	else if (temp_ptr->designated_cost > stp->root_path_cost){//非指定端口
   	  temp_ptr->designated_switch = stp->switch_id;
	  temp_ptr->designated_port = temp_ptr->port_id;
	  temp_ptr->designated_root = stp->designated_root;
	  temp_ptr->designated_cost = stp->root_path_cost;
	}
  }
}
						\end{lstlisting}
						对于非指定端口,如果节点开销小于对端节点,则转化为指定端口,需要更新上一跳节点、端口、根节点、开销等内容。对于指定端口,只需要更新根节点和开销即可。
						\item 最后是停止hello计时器和发送config消息,代码如下\\
						\begin{lstlisting}[{language={[ANSI]C}}]
//节点由根节点变为非根节点,停止计时器,但是在stp_port_send_config中已有判断
bool is_root_switch = stp_is_root_switch(stp);
if (!is_root_switch)
	stp->hello_timer.active = false;	
//将更新后的config从每个指定端口发出去
stp_send_config(stp);
						\end{lstlisting}
					\end{enumerate}
			\end{enumerate}
			\end{subsection}
		\begin{subsection}{实验结果}
			\begin{enumerate}[1)]
				\item 首先是对四个节点的环路进行测试,将stp与stp-reference的结果进行对比,如下图所示\\
				\begin{figure}[!h]	
					\centering
					\includegraphics[height=6cm]{stp_default.eps}	
					\caption{stp}
				\end{figure}	
			\begin{figure}[!h]	
				\centering
				\includegraphics[height=6cm]{reference_default.eps}	
				\caption{stp-reference}
			\end{figure}
			对比发现stp和reference的输出完全一样,表示测试成功。
				\item 然后是自己建立的拓扑图,节点关系图如图3\\
				\begin{figure*}
					\centering
					\subfigure[7个节点初始图]{
						\begin{minipage}[!h]{0.23\linewidth} \includegraphics[height=\linewidth]{7_point.eps} \end{minipage}	
						}	
					\subfigure[最终生成树图]{
						\begin{minipage}[!h]{0.23\linewidth} \includegraphics[height=\linewidth]{7_final.eps} \end{minipage}	
						}
					\caption{拓扑图}
				\end{figure*}
				\begin{figure}[!h]	
					\centering
					\includegraphics[height=9cm]{stp.eps}	
					\caption{stp}
				\end{figure}	
				\begin{figure}[!h]	
					\centering
					\includegraphics[height=8cm]{reference.eps}	
					\caption{stp-reference}
				\end{figure}
			运行reference和stp后的实验输出如图4、5\\
			对比图4、图5发现stp和reference的输出一致,且节点信息与图3(b)中的内容一致,证明建立的生成树拓扑图是正确的。
			\end{enumerate}
			\end{subsection}
		\begin{subsection}{实验总结}
			此次实验难度在于实验框架的理解,需要明确实验中各个函数的作用,设计各个函数的内容,实验中函数的流程已经在PPT中详细写出,主要难点在于如何从非指定端口中找到根端口。
		\end{subsection}
	\end{section}
\end{document}
