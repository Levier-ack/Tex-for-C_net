\documentclass{article} %article 文档
\usepackage{enumerate}
\usepackage{pgfplots}
\usepackage{listings}
\usepackage{xcolor}
\usepackage{graphicx}
\usepackage{ctex}  %使用宏包(为了能够显示汉字)
\title{计算机网络实验:第二周实验}  %文章标题
\author{秦宏 2016K8009929009}   %作者的名称
\date{\today}       %日期
% 设置页面的环境,a4纸张大小,左右上下边距信息
\usepackage[a4paper,left=10mm,right=10mm,top=15mm,bottom=15mm]{geometry}  
\pgfplotsset{compat=1.9,width=15cm}  
 
\begin{document}
\lstset{
	keywordstyle= \color{ blue!70},
	commentstyle= \color{red!50!green!50!blue!50}, 
	frame=shadowbox, % 阴影效果
	rulesepcolor= \color{ red!20!green!20!blue!20} ,
	escapeinside=``, % 英文分号中可写入中文
	xleftmargin=2em,xrightmargin=2em, aboveskip=1em
}
	\maketitle          %添加这一句才能够显示标题等信息
	\begin{section}{Socket应用编程实验}
		\begin{subsection}{实验内容}
			\begin{enumerate}[1)]
				\item 利用Socket API实现网络通信。
				\item 实现master发布报文,worker接收后分析TXT文本字母个数。
				\item 实现若干worker的接收处理,即改变配置文件即可重复任务内容。
			\end{enumerate}
			\end{subsection}
		\begin{subsection}{实验流程}
			\begin{enumerate}[1)]
				\item 编译master.c和worker.c文件。
				\item 运行topo.py文件,并根据配置文件中地址数打开端口。
				\item h1运行master程序,其余运行worker程序。
				\item 观察master程序打印出的字母个数统计。
				\item 修改workers.conf文件中地址个数和topo.py文件中worker个数,重复2)到4)。
			\end{enumerate}
			\end{subsection}
		\begin{subsection}{实验结果}
			\begin{enumerate}[1)]
				\item 三个worker下的运行结果,如图所示\\
					\begin{figure}[htb]	
						\centering
						\includegraphics[scale=0.2]{3_worker.eps}	
						\caption{三个worker下的运行结果}
					\end{figure}
				\item 三个worker下的运行结果,如图所示\\
				\begin{figure}[!htb]	
					\centering
					\includegraphics[scale=0.2]{1_worker.eps}	
					\caption{一个worker下的运行结果}
				\end{figure}
				\item 三个worker下的运行结果,如图所示\\
				\begin{figure}[htb]	
					\centering
					\includegraphics[scale=0.2]{4_worker.eps}	
					\caption{四个worker下的运行结果}
				\end{figure}
				\item 可以看到一个、三个和四个worker下的master返回得到的字母个数相同,因此实现了若干worker的任务实现。
			\end{enumerate}
		\end{subsection}
		\begin{subsection}{实验分析}
			\begin{subsubsection}{结构分析}
				\begin{enumerate}[1)]
					\item 实验任务要求读TXT的内容并对字母计数,因此首要任务是获得文件内容属性。
					\item 然后读配置文件获得需要分配的worker个数,计算得到各个worker的其实和结束位置。
					\item 基于基本的socket模型建立socket数据结构和网络连接,并生成master要发送的消息字符串,其中需要将文件名长度、起始位置和结束位置转化为字符串合并,中间采用\#号作间隔符。
					\item 采用select的方式复用I/O处理多个worker返回时的情况。
					\item 最后根据多个server\_reply的内容合并获得所有的字母总和并打印。
				\end{enumerate}
			\end{subsubsection}
			\begin{subsubsection}{代码分析}
				\begin{lstlisting}[language={[ANSI]C}]
int file_name_length = strlen(argv[1]);//获得TXT文件名
char conf_addr[50] = "/home/qin/doc/03-socket/workers.conf";//配置文件地址
char ip_addr[10][15];//存储ip地址
struct stat file_buf;//文件属性
				\end{lstlisting}
				上面定义了10个ip地址字符数组,因此该程序能分发的worker最多为10个;file\_buf通过st\_size获得文件字节数。
				\begin{lstlisting}[language={[ANSI]C}]
FILE* fp = fopen(conf_addr,"r");
if (NULL == fp){
return -1;
}

int i = 0;
while(!feof(fp)){//i为ip个数
	if(fgets(ip_addr[i],100,fp) != NULL){        
		i++;
	printf("[%d]ip is %s \n",i,ip_addr[i-1]);
	}	    
}
fclose(fp);
				\end{lstlisting}
				利用文件读取函数获得ip地址个数,用i保存。
				\begin{lstlisting}[language={[ANSI]C}]
if(stat(argv[1],&file_buf) == -1){
	return -1;
}
int handle_size = file_buf.st_size / i;//每个worker的区间

char sector_ptr[10];
itoa(sector_ptr,handle_size);
printf("file size is :%ld\n",file_buf.st_size);//整数转字符串
printf("sector length is:%s\n",sector_ptr);
				\end{lstlisting}
				计算出每个worker分配的任务区间长度,并将文件大小切换为字符串。
				\begin{lstlisting}[language={[ANSI]C}]
int sock[10];//定义10个socket
int len[10] = {0};//用于recv返回
struct sockaddr_in server[10];//对应的socket数据结构分别定义10组
char message[10][1000];
int  server_reply[10][1000];

// create socket
for(j = 0;j < i; j++){
  sock[j] = socket(AF_INET, SOCK_STREAM, 0);
  if(sock[i] == -1){
	printf("create socket[%d] failed!\n",j);
	return -1;
  }
  printf("socket[%d] created!\n",j);

  server[j].sin_addr.s_addr = inet_addr(ip_addr[j]);//分别分配ip地址
  server[j].sin_family = AF_INET;
  server[j].sin_port = htons(12345);

  if(connect(sock[j], (struct sockaddr *)&server[j], sizeof(server[j])) < 0) {
    printf("connect[%d] failed!",j);
    return 1;
  }
  printf("connected[%d]!\n",j);//建立连接

  //产生消息 = 文件名长度+‘#’+文件名+‘#’+‘起始位置’+‘#’+结束位置+'$'
  char file_name_length_ch[10];
  itoa(file_name_length_ch, file_name_length);
  strcat(message[j],file_name_length_ch);//文件名长度

  strcat(message[j], "#");

  strcat(message[j],argv[1]);

  strcat(message[j], "#");

  char start_addr[10];
  itoa(start_addr, j * handle_size);
  strcat(message[j], start_addr);

  strcat(message[j], "#");

  char end_addr[10];
  if(j == i-1)
    itoa(end_addr, file_buf.st_size);
  else
    itoa(end_addr, (j+1) * handle_size);

  strcat(message[j], end_addr);
  strcat(message[j], "$");//结束符
}
				\end{lstlisting}
				上述代码完成了连接部分和传递消息的生成。下面初始化select部分。
				\begin{lstlisting}[language={[ANSI]C}]
fd_set rfds, rfds_temp;//设置两个fd_set使得select清除之后仍然可以重新使用
FD_ZERO(&rfds);//清零操作
FD_ZERO(&rfds_temp);

for (j = 0; j < i; j++){
  FD_SET(sock[j], &rfds);//加载socket进入fd_set中
}

for (j = 0; j < i; j++){
  for(k = 0; k < 26; k++)
    server_reply[j][k] = 0;//初始化接收返回消息的数据结构
}

for (j = 0; j < i; j++){    //发送消息,只发送一次
  printf("message[%d] is %s\n",j,message[j]);
  if(send(sock[j], message[j], strlen(message[j]), 0) < 0) {
    printf("send[%d] failed!", j);
    return -1;
  }
}
				\end{lstlisting}
				上述代码完成了select初始化,下面进行select的循环部分。
				\begin{lstlisting}[language={[ANSI]C}]
while(1){

  FD_ZERO(&rfds_temp);//初始化fd_set

  rfds_temp = rfds;//获得最初的文件描述符集合

  int ready = select(FD_SETSIZE, &rfds_temp, NULL, NULL, NULL);
  //最后一个参数为NULL表示直到检测到有消息返回才继续进行
  if(ready == 0){
	printf("time out!\n");
  }
  else if(ready < 0){
	printf("select failed!\n");
  }
  else{
	printf("get socket %d!\n",ready);
  }

  for (j = 0; j < i; j++){//循环检测是否收到消息
	if(FD_ISSET(sock[j], &rfds_temp)){
	  len[j] = recv(sock[j], server_reply[j], 104, 0);
	  if (len[j] < 0) {
		printf("recv failed");
		break;
	  }
  	  server_reply[j][len[j]] = 0;
  	  m++;
  	  printf("k is %d\n",m);
 	}
  }      
  if(m == i)//当worker全部返回,跳出循环
	break;
}
// receive a reply from the server
for (k = 0; k < 26; k++){//26个字母个数分别相加并打印结果
  for(j = 1; j < i; j++)
	server_reply[0][k] += (server_reply[j][k]);
	printf("%c : %d\n", k + 'a', server_reply[0][k]);
}
				\end{lstlisting}
				上述是master.c代码部分,下面开始worker.c代码部分。
				\begin{lstlisting}[language={[ANSI]C}]
if ((s = socket(AF_INET, SOCK_STREAM, 0)) < 0) {
  perror("create socket failed");
  return -1;
}
printf("socket created\n");

// prepare the sockaddr_in structure
server.sin_family = AF_INET;
server.sin_addr.s_addr = INADDR_ANY;
server.sin_port = htons(12345);

// bind
if (bind(s,(struct sockaddr *)&server, sizeof(server)) < 0) {
  perror("bind failed");
  return -1;
}
printf("bind done\n");
fflush(stdout);
// listen
listen(s, 3);
printf("waiting for incoming connections...\n");

// accept connection from an incoming client
int c = sizeof(struct sockaddr_in);
if ((cs = accept(s, (struct sockaddr *)&client, (socklen_t *)&c)) < 0) {
  perror("accept failed");
  return -1;
}
printf("connection accepted\n");
fflush(stdout);
				\end{lstlisting}
				socket模型连接部分来自example内容,具体不再解释。下面是消息解析和处理循环。
				\begin{lstlisting}[language={[ANSI]C}]
while ((msg_len = recv(cs, msg, sizeof(msg), 0)) > 0) {

  for( i = 0; i < 26; i++)//每次处理都要初始化变量
	al_set[i] = 0;
  i = 0;
  j = 0;
  file_addr_length = 0;
  start_addr = 0;
  end_addr = 0;
//字符串解析
  do{
    file_addr_length = msg[i] - '0' + 10*file_addr_length;
    i++;
  }while(msg[i] != '#');//获取文件名长度

  while(j < file_addr_length){
	file_addr[j] = msg[i+1];
	i++;
	j++;
  }//msg[i] = "#";获取文件地址

  i++;

  do{
  	start_addr = msg[i+1] - '0' + 10*start_addr;
	i++;
  }while(msg[i+1] != '#');//获取初始位置

  i++;

  do{
	end_addr = msg[i+1] - '0' + 10*end_addr;
	i++;
  }while(msg[i+1] != '$');//获取最终位置

//文件处理函数
  CountAl(al_set, file_addr, start_addr, end_addr);

  for (j = 0; j < 26; j++){
	printf("%c:%d\n",j+'a',al_set[j]);//打印结果
	fflush(stdout);
  }
// send the message back to client
  write(cs, al_set, 104);
}
				\end{lstlisting}
				上述代码是worker执行的总体流程,下面是文件处理函数CountAl的内容。
				\begin{lstlisting}[language={[ANSI]C}]
void CountAl(int num[], char* addr, int start, int end){

/*
*num为26位整形数组,存储字母个数
*addr为文件地址
*start为起始位置,end为结束位置
*/

  char cmt_ptr[100];//每100个字符检测一组
  int temp = 0;
  int i = 0;
  int j = 0;
  int temp_r = (end - start) % 100; //求区间除以100的余数,即最后一组的元素个数
  int temp_s;//求组的个数

  if(temp_r == 0)
	temp_s = (end - start) / 100;
  else
	temp_s = (end - start) / 100 + 1;

  printf("start: %d ; end: %d\n",start,end);

  FILE* fp = fopen(addr, "r");
  if(NULL == fp)
	return;

  fseek(fp, start, SEEK_SET);

  if(temp_r == 0){
	for(i = 0; i < temp_s; i++){
	  fread(cmt_ptr, 100, 1, fp);//提取监测数据

	  for(j = 0; j < 100; j++){
	    temp = cmt_ptr[j] - 'A';

	    if(0 <= temp && temp <= 25)//不考虑大小写
		  num[temp]++;
	    else if(32 <= temp && temp <= 57)
		  num[temp - 32]++;
	  }
  	}
  }
  else{//最后一组元素不满
	for(i = 0; i < temp_s - 1; i++){
	  fread(cmt_ptr, 100, 1, fp);//提取监测数据

	  for(j = 0; j < 100; j++){
		temp = cmt_ptr[j] - 'A';

	  if(0 <= temp && temp <= 25)
		num[temp]++;
	  else if(32 <= temp && temp <= 57)
		num[temp - 32]++;
	  }
	}

    for(j = 0; j < 100; j++){//清除数组避免重复检测
  	  cmt_ptr[j] = '\0';
  	}

    fread(cmt_ptr, temp_r, 1, fp);//根据余数检测最后一组
	  for(j = 0; j < temp_r; j++){
	    temp = cmt_ptr[j] - 'A';

	    if(0 <= temp && temp <= 25)
 		  num[temp]++;
	    else if(32 <= temp && temp <= 57)
		  num[temp - 32]++;
	  }
  }
  fclose(fp);
  fp = NULL;
}
				\end{lstlisting}
				上述即是检测字母个数的函数,最终字母个数存储在num即al\_set中发送回master。
			\end{subsubsection}
		\end{subsection}
		\begin{subsection}{实验总结}
			实验难度比较大,需要考虑很多因素,难在一些细节处理的部分,包括
			\begin{enumerate}[1)]
				\item 文件处理函数要考虑分组有余数的情况检测。
				\item message解析时考虑分隔符后字符串的解析步骤。
				\item 使用select时需要保留初始的fd\_set,因为select会清除输入的fd\_set。
				\item 从配置文件中读取ip地址时的个数计数以及地址存储。
			\end{enumerate}
			总结来讲学到了很多socket编程知识,没有尝试多线程和epoll是不足之处。
		\end{subsection}
	\end{section}
\end{document}
