\documentclass{article} %article 文档
\usepackage{enumerate}
\usepackage{pgfplots}
\usepackage{listings}
\usepackage{pythonhighlight}
\usepackage{xcolor}
\usepackage{graphicx}
\usepackage{ctex}  %使用宏包(为了能够显示汉字)
\title{计算机网络实验:第三周实验}  %文章标题
\author{秦宏 2016K8009929009}   %作者的名称
\date{\today}       %日期
% 设置页面的环境,a4纸张大小,左右上下边距信息
\usepackage[a4paper,left=10mm,right=10mm,top=15mm,bottom=15mm]{geometry}  
\pgfplotsset{compat=1.9,width=15cm}  
 
\begin{document}
\lstset{
	keywordstyle= \color{ blue!70},
	commentstyle= \color{red!50!green!50!blue!50}, 
	frame=shadowbox, % 阴影效果
	rulesepcolor= \color{ red!20!green!20!blue!20} ,
	escapeinside=``, % 英文分号中可写入中文
	xleftmargin=2em,xrightmargin=2em, aboveskip=1em
}
	\maketitle          %添加这一句才能够显示标题等信息
	\begin{section}{广播网络实验}
		\begin{subsection}{实验内容}
			\begin{enumerate}[1)]
				\item 实现节点广播的broadcast\_packet函数。
				\item 验证广播网络能够正常运行。(从一个节点ping另一个节点)
				\item 验证广播网络的效率\\
				在three\_nodes\_bw.py进行iperf测量\\
				两种场景:H1:iperf client;H2,H3:servers; H1:iperf server;H2,H3:clients。
				\item 先构建环形拓扑,验证该拓扑下节点广播会产生数据包环路
			\end{enumerate}
			\end{subsection}
		\begin{subsection}{实验流程}
			\begin{enumerate}[1)]
				\item 实现节点广播
					\begin{enumerate}[(1)]
						\item 实现main.c中的broadcast\_packet函数
						\item 结果验证\\
						使用three\_nodes\_bw.py拓扑文件\\
						三个节点相互ping通。
					\end{enumerate}
				\item 广播网络传输效率\\
				实验验证广播网络的链路利用效率,进行iperf测试,H1分别做客户端和服务器。
				\item 数据包在环路中不断转发\\
				建立环形网络拓扑,由H1向H2发送一个数据包,用wireshark监听H2收到的数据包。
			\end{enumerate}
			\end{subsection}
		\begin{subsection}{实验结果}
			\begin{subsubsection}{实现节点广播}
				main.c中broad\_packet函数实现如下
				\begin{lstlisting}[language={[ANSI]C}]
void broadcast_packet(iface_info_t *iface, const char *packet, int len)
{
  iface_info_t *entry, *q;
  list_for_each_entry_safe(entry, q, &instance->iface_list, list){
    if(entry->fd != iface->fd){
	  iface_send_packet(entry, packet, len);
	  fprintf(stdout, "TODO: broadcast packet here.\n");
	}
  }
// TODO: broadcast packet 
}
				\end{lstlisting}
				编译后运行得到hub程序运行截图如下
				\begin{figure}[h]	
					\centering
					\includegraphics[width=\linewidth,scale=0.1]{hub_1.eps}	
					\caption{每个节点分别ping其余两个节点的结果}
				\end{figure}
			每个节点发出的数据包都被接收,说明三个节点均已通过hub建立连接。
			\end{subsubsection}
		\begin{subsubsection}{广播网络传输效率}
			保持main.c代码不变,H1做客户端时,H2、H3均执行iperf -s做服务器,实验截图如图2\\
			\begin{figure}[h]	
				\centering
				\includegraphics[width=\linewidth,scale=0.1]{h1_Client.eps}	
				\caption{H1做客户端}
			\end{figure}
		由上图可知H1做客户端时带宽大小接近10Mbit/s。H2的链路利用率为0.928,H3的链路利用率为0.935,H1的链路利用效率为0.9385
		。\\
		\\
		H1做服务器时,H1执行iperf -s,实验截图如图3\\
			\begin{figure}[h]	
				\centering
				\includegraphics[width=\linewidth,scale=0.1]{h1_Server.eps}	
				\caption{H1做服务器 H2为10M}
			\end{figure}
		修改H2和H3链路的带宽大小为30M,实验结果如图4\\
			\begin{figure}[h]	
				\centering
				\includegraphics[width=\linewidth,scale=0.1]{h1_Server_30.eps}	
				\caption{H1做服务器 H2为30M}
			\end{figure}
		由图3可知H1做服务器时总带宽大小为17.81Mbit/s,链路利用率为0.8955,H2的链路利用率为0.909,H3的链路利用率为0.895。
		\end{subsubsection}	
	\begin{subsubsection}{数据包在环路中不断转发}
		为了建立环形网络拓扑,需要修改tupo代码,修改如下
		\begin{python}
#!/usr/bin/python

import sys
import os.path

from mininet.topo import Topo
from mininet.net import Mininet
from mininet.link import TCLink
from mininet.cli import CLI

# Mininet will assign an IP address for each interface of a node 
# automatically, but hub or switch does not need IP address.
def clearIP(n):
for iface in n.intfList():
n.cmd('ifconfig %s 0.0.0.0' % (iface))

class BroadcastTopo(Topo):
def build(self):
h1 = self.addHost('h1')
h2 = self.addHost('h2')
b1 = self.addHost('b1')
b2 = self.addHost('b2')
b3 = self.addHost('b3')

self.addLink(h1, b1, bw=20)
self.addLink(h2, b2, bw=20)
self.addLink(b1, b2, bw=20)
self.addLink(b1, b3, bw=20)
self.addLink(b2, b3, bw=20)

if __name__ == '__main__':
if not os.path.exists('/sbin/ethtool'):
print 'ethtool not found, please install it using `apt install ethtool`'
sys.exit(1)

topo = BroadcastTopo()
net = Mininet(topo = topo, link = TCLink, controller = None) 

h1, h2, b1, b2, b3 = net.get('h1', 'h2', 'b1', 'b2', 'b3')
h1.cmd('ifconfig h1-eth0 10.0.0.1/8')
h2.cmd('ifconfig h2-eth0 10.0.0.2/8')

clearIP(b1)
clearIP(b2)
clearIP(b3)

for h in [ h1, h2, b1, b2, b3 ]:
h.cmd('./disable_offloading.sh')
h.cmd('./disable_ipv6.sh')

net.start()
# b1.cmd('./hub-reference &')
CLI(net)
net.stop()	
		\end{python}
	这样就可以建立两个主机节点,三个hub节点,编译运行得到结果如下
		\begin{figure}[h]	
			\centering
			\includegraphics[width=\linewidth,scale=0.1]{bw_3_1.eps}	
			\caption{建立环形拓扑}
		\end{figure}
	在H1中向H2发送数据包,用wireshark监听结果如图5所示
			\begin{figure}[h]	
			\centering
			\includegraphics[width=\linewidth,scale=0.1]{bw_3_2.eps}	
			\caption{wireshark监测结果}
		\end{figure}
	\end{subsubsection}
		\end{subsection}
		\begin{subsection}{实验分析}
			\begin{enumerate}[1)]
				\item 广播网络传输效率分析\\
				H1分别处于客户端状态和服务器状态下,数据包的传输带宽保持在接近10M的大小,近似认为该次数据包传输的带宽上限是10M,无论数据包的传输方向是从H1到H2还是H2到H1,传输通路的带宽上限与通路中最小带宽的上限一致。图3这里就是10M,图4这里就是20M
				\item 数据包在环路中的传输\\
				如图出现UDP提示,表示收到多个重复数据包,表明数据包多次发送,根据wireshark的监测结果,当H1未发包时,H2未收到数据包,发包后,H2处wireshark的监测结果显示H2处不断接收到数据包,表明三个hub之间形成的环路拓扑导致数据包不断广播传输,使得H2能不断接收到数据包。
			\end{enumerate}
		\end{subsection}
		\begin{subsection}{实验总结}
			此次实验难度不大,原因是main.c的大部分代码已经实现,只需要按步骤做实验即可。
		\end{subsection}
	\end{section}
\end{document}
